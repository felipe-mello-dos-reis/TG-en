Cement is one of the main materials in civil construction, being used from the construction of houses and buildings to bridges and highways.
In developing countries such as Brazil, cement is widely used due to its low complexity and cost, which allows its use on a large scale in any location.
The exponential increase in cement production, 10 times greater than the world population growth \cite{united1995world}, has been accompanied by a significant share of greenhouse gas (GHG) emissions, due to the calcination process of limestone that transforms calcium carbonate into calcium oxide and carbon dioxide in high-temperature furnaces.
The production of Portland cement generates on average 842 kg of $CO_2/t$ of clinker produced \cite{andrew2018global}, representing 5\% of anthropogenic GHG emissions \cite{IEA_WBCSD_2009}.

In this context, there is a need to develop new cementitious materials that present three main properties: low GHG emissions, low cost, and high strength/durability \cite{scrivener2018eco}.

Alkaline-activated materials (AAM) - solid precursors rich in silica ($SiO_2$) and alumina ($Al_2O_3$), capable of forming binding gels composed of sodium-alumino-silicate hydrate (N-A-S-H) and calcium-alumino-silicate hydrate (C-A-S-H) - have gained prominence due to their potential to partially or totally replace Portland cement, significantly reducing the GHG emissions associated with conventional cement production.

There are two ways in which AAM can be produced: by mixing the solid precursor with a liquid alkaline activator or with a solid alkaline source and water.
Two-part systems have been widely employed in the initial development of this market due to their high mechanical performance, durability, and chemical resistance.
However, one-part systems are a more scalable technology due to the lower risk of handling and storing of solid activators \cite{provis2018alkali}.

Calcium-rich solid precursors are widely used due to several advantages, such as rapid strength development \cite{provis2014geopolymers}, reduced reliance on thermal curing \cite{ke2021one}, and the formation of denser and less porous matrices by C-A-S-H reaction products compared to N-A-S-H gels \cite{bernal2014engineering}.
In this context, there remains a technical and scientific gap in the formulation and characterization of low-calcium AAMs using alternative alkaline sources, as most studies focus on blast furnace slag as a precursor and rely on hydroxides and silicates as activators \cite{zareechian2023advancements}.

This work proposes the development of a one-part alkaline-activated cement focusing on low-calcium solid precursors, specifically metakaolin and silica fume, combined with safer and more accessible alternative alkaline sources, such as potassium carbonate ($K_2CO_3$) and calcium hydroxide ($Ca(OH)_2$).
This approach aims to contribute to the formulation of more sustainable, safe, and adequately performing binders for applications in civil construction, aligning with contemporary guidelines for low environmental impact \cite{PNRS2016}.