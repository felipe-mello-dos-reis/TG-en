Portland cement is one of the main materials in civil construction, being used from the construction of houses and buildings to bridges and highways.
In developing countries such as Brazil, cement is widely used due to its low complexity and cost, which allows its use on a large scale in any location.
The exponential increase in cement production, having more than 30-fold since 1950 and almost 4-fold since 1990 \cite{usgs2016cement}, has been accompanied by a significant share of greenhouse gas (GHG) emissions, due to the calcination process of limestone that transforms calcium carbonate into calcium oxide and carbon dioxide in high-temperature furnaces.
The production of Portland cement generates on average 842 kg of $CO_2/t$ of clinker produced \cite{andrew2018global}, representing 5.6\% of anthropogenic GHG emissions \cite{le2018global}.

In this context, there is a need to develop new cementitious materials that present three main properties: low GHG emissions, low cost, and high strength/durability \cite{scrivener2018eco}.
A promising approach involves the implementation of supplementary cementitious materials (SCMs) or the introduction of new types of cements such as alkali-activated materials (AAM) which are produced by a mixture between a solid precursor and an alkali source.
These materials can partially or totally replace ordinary Portland cement by incorporating industrial products rich in silica and alumina, such as metakaolin, silica fume, or fly ash, significantly reducing the environmental footprint while maintaining or even enhancing mechanical performance and durability.

There are two ways in which AAM can be produced: by mixing the solid precursor with a liquid alkaline activator or with a solid alkaline source and water.
Traditional two-part formulations, which require liquid activators such as sodium silicate or hydroxide, achieve high mechanical performance but are difficult to handle and store safely.
One-part binders, in contrast, only require the addition of water to trigger the activation reaction, significantly improving scalability, storage safety, and field applicability \cite{provis2018alkali}.
This approach also reduces the environmental and economic burdens associated with liquid activators, while maintaining satisfactory mechanical and durability performance under optimized mix conditions.

Despite the progress in AAM, most current research relies on conventional activators - particularly silicates and hydroxides \cite{zareechian2023advancements} - that pose environmental, economic, and safety challenges.
The production of sodium silicate, for instance, emits about 1.2 kg of CO\textsubscript{2} per kg produced \cite{neupane2022evaluation} and requires temperatures above 1200-1400 $\degree$C \cite{vinai2019production}, while hydroxide solutions are highly caustic and hazardous for large-scale handling.
The substitution of these activators by solid, less aggressive alternatives such as potassium carbonate ($K_2CO_3$) or calcium hydroxide ($Ca(OH)_2$) can mitigate these issues by reducing energy demand, minimizing GHG emissions, and improving user safety.

Furthermore, calcium-rich precursors such as blast furnace slag and fly ash have dominated alkali-activation research due to several advantages, such as rapid strength development \cite{provis2014geopolymers}, reduced reliance on thermal curing \cite{ke2021one}, and the formation of denser and less porous matrices by C-A-S-H reaction products compared to N-A-S-H gels \cite{bernal2014engineering}.
However, their availability is regionally limited and their chemistry often results in brittle matrices prone to shrinkage and cracking. Low-calcium precursors—such as metakaolin and silica fume—offer a broader, more sustainable alternative, forming amorphous N-A-S-H type gels with low permeability, high thermal stability, and reduced susceptibility to chemical attack.

Expandig the use of these materials is particularly relevant for one-part systems, as they can incorporate a wider variety of industrial by-products and natural minerals while maintaining durability and reducing environmental impact.
In this sense, this work proposes the development of one-part alkali-activated cements using low-calcium solid precursors and alternative alkaline sources.
This approach aims to contribute to the formulation of more sustainable, safe, and adequately performing binders for applications in civil construction, aligning with contemporary guidelines for low environmental impact \cite{PNRS2016}.