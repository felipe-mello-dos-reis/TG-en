\section{Materials and Characterization}
\label{sec:materials_and_characterization}

For the development of one-part geopolymeric mortars, the following components were used:

\begin{itemize}
    \item Metakaolin (MK) supplied by the company \textcolor{red}{XXX};
    \item Silica fume (SF) supplied by the company \textcolor{red}{XXX};
    \item Potassium carbonate supplied by the company \textcolor{red}{XXX};
    \item Calcium hydroxide supplied by the company \textcolor{red}{XXX};
    \item Standardized quartz sand supplied by the company \textcolor{red}{XXX};
    \item Distilled water.
\end{itemize}

The binders used were silica fume and metakaolin. However, the purity of commercially available metakaolin is not sufficient to ensure precision in the characterization of cementitious samples.
Therefore, it was produced from commercial kaolin, as detailed in Section \ref{sec:production_of_metakaolin}.
The alkaline sources are commercially available with high purity, so the physicochemical composition provided by the manufacturer was used.

The chemical composition of the materials used in the formulation of the mortars is presented in Table \ref{tab:chemical_composition_reagents}.

\begin{table}[H]
    \caption{Chemical properties of the solid reagents.}
    \label{tab:chemical_composition_reagents}
    \center
    \begin{tabular}{ccc}
        \hline
        Material & Chemical composition & Specification (\%)\\
        \hline
        Silica fume & $SiO_2$ &  \textcolor{red}{XX,X} \\
            & $ Al_2O_3$ & \textcolor{red}{XX,X} \\
            & $MgO$ & \textcolor{red}{XX,X} \\
            & $CaO$ & \textcolor{red}{XX,X} \\
            & $Fe_2O_3$ & \textcolor{red}{XX,X} \\
        Potassium carbonate & $K_2CO_3$ & \textcolor{red}{XX,X} \\
        Calcium hydroxide & $Ca(OH)_2$ & \textcolor{red}{XX,X} \\
        \hline
    \end{tabular}
\end{table}

In addition, the quartz sand used follows the standards established by the Institute for Technological Research (IPT), as shown in Tables \ref{tab:quartz_sand_properties} and \ref{tab:quartz_sand_granulometry}.

\begin{table}[H]
    \caption{Results of physical and chemical requirements of standardized quartz sand.}
    \label{tab:quartz_sand_properties}
    \center
    \begin{tabular}{p{0.30\textwidth} p{0.25\textwidth} p{0.25\textwidth}}
        \hline
        Property & Result & ABNT NBR7214:2015 Requirement\\
        \hline
        Silica content (ABNT NBR14656:2001) & 96.5\% & $\geq$ 95\%, by mass \\
        Moisture (ABNT NBR7214:2015) & 0.0\% & $\leq$ 0.2\%, by mass \\
        Organic matter (ABNT NBR17053:2022) & Lighter or equal to the color of the standard solution & Color of the 2\% tannic acid standard solution \\
        \hline
    \end{tabular}
\end{table}

\begin{table}[H]
    \caption{Particle size distribution of the fractions of standardized quartz sand.}
    \label{tab:quartz_sand_granulometry}
    \centering
    \begin{tabular}{p{0.10\textwidth} p{0.30\textwidth} p{0.15\textwidth} p{0.25\textwidth}}
        \hline
        \multirow{2}{*}{Fraction} & \multirow{2}{*}{Sieve interval} & \multicolumn{2}{c}{Mass percentage (\%)} \\ \cline{3-4}       
        & & Result & ABNT NBR7214:2015 Requirement\\
        \hline
        16 & (2.4 mm and 2.0 mm) & 0 & $\leq$ 10 \\
        16 & (2.0 mm and 1.2 mm) & 97 & $\geq$ 90 \\
        30 & (1.2 mm and 0.6 mm) & 99 & $\geq$ 95 \\
        50 & (0.6 mm and 0.3 mm) & 96 & $\geq$ 95 \\
        100 & (0.3 mm and 0.15 mm) & 95 & $\geq$ 95 \\
        \hline
    \end{tabular}
\end{table}

\section{Production of Metakaolin}
\label{sec:production_of_metakaolin}

Metakaolin was obtained by calcining kaolin at $700\ ^\circ C$ for 1 hour in an electric furnace from \textcolor{red}{XXX}.
The optimal calcination time and temperature were determined from preliminary tests, where the yield of calcination was evaluated.
To ensure material homogeneity, two shallow trays with a maximum height of \textcolor{red}{XX} mm were used.
The transformation of crystalline kaolinite into amorphous metakaolinite is represented by Equation \ref{eq:calcination_kaolin}.

\begin{equation}
    \label{eq:calcination_kaolin}
        Al_2.2Si_2O_2.2H_2O \xrightarrow{\Delta} Al_2O_3.2SiO_2 + 2 H_2O
\end{equation}

\section{Physicochemical Characterization of Solid Precursors}
\label{sec:physicochemical_characterization_precursors}

The physicochemical characterization of the solid precursors was carried out at the laboratories of the Aeronautics Institute of Technology (ITA), located in São José dos Campos-SP.

Energy-dispersive X-ray spectroscopy (EDS) was used to determine the proportion of chemical elements, performed together with scanning electron microscopy (SEM) to evaluate the morphology of the solid precursors.

In addition, X-ray diffraction (XRD) was used to determine the crystalline phase of metakaolin and silica fume.

To verify the removal of hydroxyl groups ($OH^-$) and the presence of $Al-Si-O$ bonds, Fourier-transform infrared spectroscopy (FTIR) was performed using a spectrometer from \textcolor{red}{XXX}.

% As distribuições granulométricas do metacaulim (MC), escória de alto forno (EAF) e sílica ativa (SA) foram determinadas por difração a LASER, utilizando-se um equipamento CILAS1090. As amostras foram dispersas em água destilada, e as condições de ensaio adotadas incluíram agitação a 1500 rpm, tempo de ultrassom de 2,5 minutos, obscuração entre 10 e 20%, e tempo total de dispersão de 5 minutos.

Finally, the particle size distribution of the solids used was determined by laser granulometry. Smaller and more irregular particles tend to have a higher specific surface area and, therefore, greater reactivity in contact with the alkaline source, which can directly influence the mechanical performance and rheological properties of the geopolymeric mortars.

\section{Production of Geopolymeric Mortars and Pastes}
\label{sec:production_geopolymeric_mortars_pastes}

\subsection{Mix Design}
\label{subsec:mix_design}

The development of one-part geopolymeric mortars followed a systematic experimental design, aiming to evaluate the effect of different compositions on physicochemical and mechanical properties.

The variables considered in the study were:

\begin{itemize}
    \item Proportion between solid precursors (metakaolin and silica fume);
    \item Content of alkaline activators ($K_2CO_3$ and $Ca(OH)_2$);
    \item Water/solids ratio (w/s);
    \item Sand/binder ratio (s/b).
\end{itemize}

The variable of interest in this experiment is the $Si/Al$ ratio, which will be varied from 1.0 to 5.0, calculated based on the proportion of metakaolin and silica fume.

Initially, the water/solids ratio was determined analogously to \textcolor{red}{ref xxxx}, to ensure adequate workability of the pastes.

In addition, due to stoichiometric balance, the $Al/K$ ratio will be constant and equal to 1, according to the empirical formula \ref{eq:al_k_ratio} \cite{joseph1991geopolymers}, where $M$ is a sodium or potassium cation.

\begin{equation}
    \label{eq:al_k_ratio}
    M_n \left\{ \left(SiO_2 \right)_z AlO_2 \right\}_n \cdot wH_2O
\end{equation}

Furthermore, the $K/Ca$ ratio will be constant and equal to 2, respecting the precipitation reaction of potassium carbonate with calcium hydroxide, as shown in Equation \ref{eq:k_ca_reaction}.

\begin{equation}
    \label{eq:k_ca_reaction}
    K_2CO_3 + Ca(OH)_2 \rightarrow  2KOH_{(aq)} + CaCO_{3(s)} \downarrow
\end{equation}

With the paste proportions well defined, the production of mortars maintained a 1:3 ratio between binder and sand, as recommended in the literature \textcolor{red}{ref XXXX}.

Table \ref{tab:mortar_compositions} presents the different formulations produced, with the respective mass proportions of the components.

\begin{table}[H]
    \caption{Compositions of the produced geopolymeric mortars.}
    \label{tab:mortar_compositions}
    \center
    \begin{tabular}{cccccccc}
    \hline
    Sample & \multicolumn{2}{c}{Precursors (\%)} & \multicolumn{2}{c}{Activators (\%)} & \multirow{2}{*}{w/s} & \multirow{2}{*}{s/b} & \multirow{2}{*}{Si/Al} \\
    \cline{2-5}
     & MK & SF & $K_2CO_3$ & $Ca(OH)_2$ & & & \\
    \hline
    A1 & \textcolor{red}{XX} & \textcolor{red}{XX} & \textcolor{red}{X,X} & \textcolor{red}{X,X} & \textcolor{red}{X,XX} & \textcolor{red}{X,X} & \textcolor{red}{1,0} \\
    A2 & \textcolor{red}{XX} & \textcolor{red}{XX} & \textcolor{red}{X,X} & \textcolor{red}{X,X} & \textcolor{red}{X,XX} & \textcolor{red}{X,X} & \textcolor{red}{2,0} \\
    A3 & \textcolor{red}{XX} & \textcolor{red}{XX} & \textcolor{red}{X,X} & \textcolor{red}{X,X} & \textcolor{red}{X,XX} & \textcolor{red}{X,X} & \textcolor{red}{3,0} \\
    A4 & \textcolor{red}{XX} & \textcolor{red}{XX} & \textcolor{red}{X,X} & \textcolor{red}{X,X} & \textcolor{red}{X,XX} & \textcolor{red}{X,X} & \textcolor{red}{4,0} \\
    A5 & \textcolor{red}{XX} & \textcolor{red}{XX} & \textcolor{red}{X,X} & \textcolor{red}{X,X} & \textcolor{red}{X,XX} & \textcolor{red}{X,X} & \textcolor{red}{5,0} \\
    \hline
    \end{tabular}
\end{table}

\subsection{Mixing Procedure}
\label{subsec:mixing_procedure}

The production of the mixtures followed standardized procedures. For the production of mortars and compressive strength testing, the procedures of the Brazilian standard \cite{ABNT_NBR_7215_2019} were followed. For the production of pastes, the American standard \cite{ASTM_C305_2006} was chosen, since the Brazilian standard does not specify the mixing procedure for cementitious pastes without fine aggregate. Both procedures were adapted for the preparation of small-volume samples.

\subsection{Molding and Curing of Specimens}
\label{subsec:molding_and_curing_specimens}

For the compressive strength test, the specimens were prepared in prismatic molds with dimensions \textcolor{red}{XX} × \textcolor{red}{XX} × \textcolor{red}{XX} mm, previously lubricated with oil-based release agent.

For each composition, 9 specimens were molded, intended for testing at ages of 1, 3, and 7 days (3 specimens for each age). It was not necessary to perform tests at 28 days, as the thermal curing of the binders used provides high initial strength gain, as demonstrated in the literature \textcolor{red}{ref XXX}.

Thermal curing was carried out in an oven maintained at (60 ± 2)°C and a minimum relative humidity of 95\% for 24 hours, as recommended by the standard \cite{ABNT_NBR_9479_2006}, to ensure the activation of the binders and accelerate the curing process.
It is noteworthy that demolding was performed 24 hours after the start of the curing process.

Demolding was performed 24 hours after molding, and the specimens were immediately transferred to the corresponding curing conditions until the testing age.

For microstructural analyses, small samples were separated, with hydration interrupted by immersion in ethyl alcohol and vacuum filtration, followed by drying in an oven at 40°C for 24 hours.
%% EXPLICAR PQ ALcool
% The alcohol was used to prevent rehydration of the samples, as it is a dehydrating agent that does not react with the components of the geopolymeric paste.
These samples were stored in hermetically sealed containers to prevent rehydration.

\section{Characterization of Geopolymeric Mixtures}
\label{sec:characterization_geopolymeric_mixtures}

\subsection{Hardened State Tests}
\label{subsec:hardened_state_tests}

\subsubsection{Compressive Strength}
\label{subsubsec:compressive_strength}

For the compressive strength test, the specimens were placed in a hydraulic press from \textcolor{red}{XXX}, model \textcolor{red}{XXX}, applying load at a rate of \textcolor{red}{XXX} N/s until failure. The strength was calculated by the equation:

\begin{equation}
    \label{eq:compressive_strength}
    R_c = \frac{F_c}{A_t}
\end{equation}

Where:
\begin{itemize}
    \item $R_c$ is the compressive strength, in MPa;
    \item $F_c$ is the maximum applied load, in N;
    \item $A_t$ is the cross-sectional area, in mm\textsuperscript{2}.
\end{itemize}

For statistical analysis, the Tukey test was performed, allowing the identification of significant differences between sample groups, considering a significance level of \textcolor{red}{XXX}\%.

\subsection{Microstructural Analyses}
\label{subsec:microstructural_analyses}

\subsubsection{X-ray Diffraction (XRD)}
\label{subsubsec:xrd}

X-ray diffraction analyses were performed on ground paste samples with particle size less than 75 µm, at ages of 7 and 28 days. A diffractometer from \textcolor{red}{XXX}, model \textcolor{red}{XXX}, was used.
%  with copper radiation (Cu-Kα, λ = 1.5418 Å), operating at 40 kV and 30 mA. Scans were performed at an angular speed of 0.02° per second, in a range from 5° to 80° (2θ).

\subsubsection{Fourier Transform Infrared Spectroscopy (FTIR)}
\label{subsubsec:ftir}

Infrared spectroscopy was performed on ground paste samples with particle size less than 45 µm, at ages of 1, 3, and 7 days. A spectrometer from \textcolor{red}{XXX}, model \textcolor{red}{XXX}, was used, in the range of \textcolor{red}{XXX} to \textcolor{red}{XXX} cm$^{-1}$, with a resolution of \textcolor{red}{XXX} cm$^{-1}$ and \textcolor{red}{XXX} scans.

\subsubsection{Scanning Electron Microscopy (SEM)}
\label{subsubsec:sem}

The microstructural analysis of the pastes was performed by scanning electron microscopy, using a microscope from \textcolor{red}{XXX}, model \textcolor{red}{XXX}, coupled to an energy-dispersive X-ray spectrometer (EDS). The samples were prepared from cylindrical fragments of pastes molded in a disposable plastic straw.
% Analyses were performed at magnifications of 500×, 2000×, and 5000×, with an acceleration voltage of 15 kV.

