\section{Materials}
\label{sec:materials}

For the development of one-part alkali activated pastes and mortars, the following components were used:

\begin{itemize}
    \item Kaolin supplied by the company Brasilminas;
    \item Silica fume (SF) supplied by the company Elken;
    \item Potassium carbonate supplied by the company Neon (purity of 98\%);
    \item Calcium hydroxide supplied by the company Neon  (purity of 95\%);
    \item Standardized quartz sand supplied by the company Instituto de Pesquisas Tecnológicas (IPT);
    \item Distilled water.
\end{itemize}

% However, the purity of commercially available metakaolin is not sufficient to ensure precision in the characterization of cementitious samples.
The precursors used were silica fume and metakaolin - which  was produced from commercial kaolin, as detailed in Section \ref{sec:production_of_metakaolin}.
The alkaline sources are commercially available with high purity, so the physicochemical compositions provided by the manufacturer were used.

In addition, the quartz sand used follows the standards established by the Institute for Technological Research (IPT), as shown in Tables \ref{tab:quartz_sand_properties} and \ref{tab:quartz_sand_granulometry}.

\begin{table}[H]
    \caption{Results of physical and chemical requirements of standardized quartz sand.}
    \label{tab:quartz_sand_properties}
    \center
    \begin{tabular}{p{0.30\textwidth} p{0.25\textwidth} p{0.25\textwidth}}
        \hline
        Property & Result & ABNT NBR7214:2015 Requirement\\
        \hline
        Silica content (ABNT NBR14656:2001) & 96.5\% & $\geq$ 95\%, by mass \\
        Moisture (ABNT NBR7214:2015) & 0.0\% & $\leq$ 0.2\%, by mass \\
        Organic matter (ABNT NBR17053:2022) & Lighter or equal to the color of the standard solution & Color of the 2\% tannic acid standard solution \\
        \hline
    \end{tabular}
\end{table}

\begin{table}[H]
    \caption{Particle size distribution of the fractions of standardized quartz sand.}
    \label{tab:quartz_sand_granulometry}
    \centering
    \begin{tabular}{p{0.10\textwidth} p{0.30\textwidth} p{0.15\textwidth} p{0.25\textwidth}}
        \hline
        \multirow{2}{*}{Fraction} & \multirow{2}{*}{Sieve interval} & \multicolumn{2}{c}{Mass percentage (\%)} \\ \cline{3-4}       
        & & Result & ABNT NBR7214:2015 Requirement\\
        \hline
        16 & (2.4 mm and 2.0 mm) & 0 & $\leq$ 10 \\
        16 & (2.0 mm and 1.2 mm) & 97 & $\geq$ 90 \\
        30 & (1.2 mm and 0.6 mm) & 99 & $\geq$ 95 \\
        50 & (0.6 mm and 0.3 mm) & 96 & $\geq$ 95 \\
        100 & (0.3 mm and 0.15 mm) & 95 & $\geq$ 95 \\
        \hline
    \end{tabular}
\end{table}

\section{Methods}
\label{sec:methods}
The experimental process was divided into three steps: (i) production and characterization of raw-materials; (ii) production of geopolymeric pastes and mortars; (iii) microstructural studies and physical properties of hardened state.

\subsection{Production and Characterization of Raw Materials}
\label{sec:production_characterization_raw_materials}

\subsubsection{Production of Metakaolin}
\label{sec:production_of_metakaolin}

Metakaolin was obtained by calcining kaolin at $700\ ^\circ C$ for 1 hour in a 200 L 18 kW laboratory furnace.
The optimal calcination time and temperature were determined from preliminary tests, in which the yield of calcination and reactivity was evaluated.
To ensure material homogeneity, two shallow trays with a maximum height of 10 mm were used.
The transformation of crystalline kaolinite into amorphous metakaolinite is represented by Equation \ref{eq:calcination_kaolin}.

\begin{equation}
    \label{eq:calcination_kaolin}
        Al_2.Si_2O_5\left(OH\right)_4 \xrightarrow{\Delta} Al_2O_3.2SiO_2 + 2 H_2O
\end{equation}

\subsubsection{Physicochemical Characterization of Solid Precursors}
\label{sec:physicochemical_characterization_precursors}

The physicochemical characterization of the solid precursors was carried out at the laboratories of the Aeronautics Institute of Technology (ITA), located in São José dos Campos-SP.

The oxide compositions of metakaolin and silica fume were determined by X-ray Fluorescence (XRF).
To complement this XRF analysis, the removal of hydroxyl groups ($OH^-$) was verified by loss on ignition (LOI).
The results were used to calculate the Si/Al molar ratios and to verify the purity of the precursors.

% Furthermore, the particle size distribution (PSD) of the solids used was determined by laser diffraction. Smaller and more irregular particles tend to have a higher specific surface area and, therefore, greater reactivity in contact with the alkaline source, which can directly influence the mechanical performance and rheological properties of the geopolymeric mortars.
% PSD was analyzed in a Malvern Mastersizer 3000 particle size analyzer, with air as the dispersion agent, at 1.5 bar pressure and a 40\% feed rate.

In addition, X-ray diffraction (XRD) was used to verify the absence crystalline phases of metakaolin and silica fume, as they are amorphous.
XRD was performed by a Panalytical Empyrean diffractometer, with a $2\theta$ interval of $10$-$70^\circ$, Cu-$\mathrm{K}\alpha$ radiation, $0.01^\circ$ step and 50 s/step.


% As distribuições granulométricas do metacaulim (MC), escória de alto forno (EAF) e sílica ativa (SA) foram determinadas por difração a LASER, utilizando-se um equipamento CILAS1090. As amostras foram dispersas em água destilada, e as condições de ensaio adotadas incluíram agitação a 1500 rpm, tempo de ultrassom de 2,5 minutos, obscuração entre 10 e 20%, e tempo total de dispersão de 5 minutos.
To verify the presence of bonds, Fourier-transform infrared spectroscopy (FTIR) was carried out in a PerkinElmer spectrophotometer, with a spectrum range of $4000$-$400\ cm^{-1}$ and a spectral resolution of $1\ cm^{-1}$.

Finally, Energy-dispersive X-ray spectroscopy (EDS) was performed together with scanning electron microscopy (SEM) to evaluate the morphology of the solid precursors.
SEM/EDS was conducted using a TESCAN VEGA 3 XMU device and Oxford EDS 133 eV detector, with gold coated.


\subsection{Production of Geopolymeric Pastes and Mortars}
\label{sec:production_geopolymeric_pastes_mortars}

\subsubsection{Mix Design}
\label{sec:mix_design}

The development of one-part geopolymeric mortars followed a systematic experimental design, aiming to evaluate the effect of different compositions on physicochemical and mechanical properties.

The variable of interest in this experiment is the $Si/Al$ ratio, which will be varied from 1.0 to 5.0, calculated based on the proportion of metakaolin and silica fume.

The following proportions were considered constant in this study:

\begin{itemize}
    \item Mass ratio water/solids (w/s);
    \item Molar ratio Al/K;
    \item Molar ratio K/Ca;
    \item Mass ratio sand/binder for mortars (s/b).
\end{itemize}

Initially, the water-to-solids ratio was determined to ensure adequate workability of the pastes, targeting a spread diameter of 200–250 mm in the mini-slump test.

In addition, due to stoichiometric balance, the $Al/K$ ratio will be constant and equal to 1, according to the empirical formula on Equation \ref{eq:al_k_ratio} \cite{joseph1991geopolymers}, where $M$ is a sodium or potassium cation.

\begin{equation}
    \label{eq:al_k_ratio}
    M_n \left\{ \left(SiO_2 \right)_z AlO_2 \right\}_n \cdot wH_2O
\end{equation}

Furthermore, the $K/Ca$ ratio will be constant and equal to 2, respecting the precipitation reaction of potassium carbonate with calcium hydroxide, as shown in Equation \ref{eq:k_ca_reaction}.

\begin{equation}
    \label{eq:k_ca_reaction}
    K_2CO_3 + Ca(OH)_2 \rightarrow  2KOH_{(aq)} + CaCO_{3(s)} \downarrow
\end{equation}

With the paste proportions well defined, the production of mortars maintained a 1:2 ratio between binder and sand, analogously in the previous researches \cite{batista2025mgosio2} and \cite{arellano2014geopolymer}. 

\subsubsection{Mixing Procedure}
\label{sec:mixing_procedure}

The production of the mixtures followed standardized procedures. For the production of mortars and compressive strength testing, the procedures of the European standard \cite{EN1961_2016} were followed. For the production of pastes, the American standard \cite{ASTM_C305_2006} was chosen, since the Brazilian standard does not specify the mixing procedure for cementitious pastes without fine aggregate. Both procedures were adapted for the preparation of small-volume samples.

\subsubsection{Molding and Curing of Specimens}
\label{sec:molding_and_curing_specimens}

For the compressive strength test, the specimens were prepared in prismatic molds with dimensions $40 \times 40 \times 40 \ mm$, previously lubricated with oil-based release agent.

For each composition, 9 specimens were molded, intended for testing at ages of 1 and 3 days (3 specimens for each age). It was not necessary to perform tests at longer curing ages, as the thermal curing of the binders used provides high initial strength gain \cite{aredes2015effect}.

Thermal curing was carried out in an oven maintained at (60 ± 2)°C and a minimum relative humidity of 95\% for 24 hours, as recommended by the standard \cite{ABNT_NBR_9479_2006}, to ensure the activation of the binders and accelerate the curing process.
By curing at this temperature it is possible to minimize the pore volume, therefore achieve higher compressive strength \cite{aredes2015effect}.
It is noteworthy that demolding was performed 24 hours after the start of the curing process. After demolding, the specimens were immediately transferred to the corresponding curing conditions until the testing age.

%% EXPLICAR PQ ALcool
% The alcohol was used to prevent rehydration of the samples, as it is a dehydrating agent that does not react with the components of the geopolymeric paste.
These samples were stored in hermetically sealed containers to prevent rehydration.

\subsection{Microstructural studies and physical properties of hardened state}
\label{sec:microstructural_studies_physical_properties_hardened_state}

\subsubsection{Characterization of Geopolymeric pastes}
\label{sec:characterization_geopolymeric_pastes}

The microstructural characterization of the hardened pastes was performed by X-ray diffraction (XRD), Fourier-transform infrared spectroscopy (FTIR), mercury intrusion porosimetry (MIP) and scanning electron microscopy (SEM) with energy-dispersive X-ray spectroscopy (EDS). 
Moreover, XRD was performed under the same conditions applied for the raw materials.
Moreover, XRD and FTIR were performed under the same conditions applied for the raw materials.

For XRD analysis, small samples were grinded, then, the polymerization was interrupted by immersion in ethyl alcohol - as it is a dehydrating agent that does not react with the components of the geopolymeric paste- and vacuum filtration, followed by drying in an oven at 50$\degree$C for 24 hours.


\subsubsection{Physical properties of mortars}
\label{sec:physical_properties_mortars}

Water absorption and apparent density tests were conducted on the hardened mortars to evaluate their porosity-related properties. These tests followed the procedures established by the ASTM C642–13 standard \cite{ASTM_C642_2013} which provides guidelines for determining the volume of permeable voids through oven-drying, immersion, and weighting steps. 

Additionally, for the compressive strength test, the specimens were placed in a hydraulic press with  a 600 kN-load limit, applying load at a rate of 0.5 kN/s until failure. The strength was calculated by the equation:

\begin{equation}
    \label{eq:compressive_strength}
    R_c = \frac{F_c}{A_t}
\end{equation}

Where:
\begin{itemize}
    \item $R_c$ is the compressive strength, in MPa;
    \item $F_c$ is the maximum applied load, in N;
    \item $A_t$ is the cross-sectional area, in mm\textsuperscript{2}.
\end{itemize}

For statistical analysis, the Tukey test was performed, allowing the identification of significant differences between sample groups, considering a significance level of 95\%.
The Table \ref{tab:tests_summary} summary presents the summary of the methods used in this work.

\begin{table}[H]
    \centering
    \caption{Summary of tests performed, samples used, curing conditions and main objetives.}
    \label{tab:tests_summary}
    \begin{tabular}{p{2.5cm} p{2cm} p{2.5cm} p{2.5cm} p{4.5cm}}
    \hline
        Test & Precursors & Si/Al ratios & Curing days & Objetive \\
        \hline
        XRF & Yes & -- & -- & Determination of oxide content to calculate molar ratios \\
        LOI & Yes & -- & -- & Assessment of hydroxyl removal and quantification of volatile matter \\
        % PSD & No & 0.9, 3.0, 5.0 & 3 & Evaluation of granulometry and specific surface area. \\
        XRD & Yes & 0.9 - 5.0 & 3 & Identification of crystalline and amorphous phases \\
        FTIR & Yes & 0.9 - 5.0 & 1, 3 & Identification of chemical bonds in order to monitor the geopolymerization \\
        SEM/EDS & No & 0.9, 3.0, 5.0 & 3 & Observation of morphology and elemental distribution \\
        MIP & No & 0.9, 3.0, 5.0 & 3 & Determination of pore size distribution and total porosity \\
        Porosity & No & 0.9 - 5.0 & 3 & Measurement water absorption and bulk density \\
        Compressive strength & No & 0.9 - 5.0 & 1, 3 & Evaluation of mechanical performance \\
        \hline
    \end{tabular}
\end{table}

% \subsection{Microstructural Analyses}
% \label{sec:microstructural_analyses}

% \subsubsection{X-ray Diffraction (XRD)}
% \label{subsec:xrd}

% X-ray diffraction analyses were performed on ground paste samples with particle size less than 75 µm, at ages of 7 and 28 days. A diffractometer from \textcolor{red}{XXX}, model \textcolor{red}{XXX}, was used.
% %  with copper radiation (Cu-Kα, λ = 1.5418 Å), operating at 40 kV and 30 mA. Scans were performed at an angular speed of 0.02° per second, in a range from 5° to 80° (2θ).

% \subsubsection{Fourier Transform Infrared Spectroscopy (FTIR)}
% \label{subsec:ftir}

% Infrared spectroscopy was performed on ground paste samples with particle size less than 45 µm, at ages of 1, 3, and 7 days. A spectrometer from \textcolor{red}{XXX}, model \textcolor{red}{XXX}, was used, in the range of \textcolor{red}{XXX} to \textcolor{red}{XXX} cm$^{-1}$, with a resolution of \textcolor{red}{XXX} cm$^{-1}$ and \textcolor{red}{XXX} scans.

% \subsubsection{Scanning Electron Microscopy (SEM)}
% \label{subsec:sem}

% The microstructural analysis of the pastes was performed by scanning electron microscopy, using a microscope from \textcolor{red}{XXX}, model \textcolor{red}{XXX}, coupled to an energy-dispersive X-ray spectrometer (EDS). The samples were prepared from cylindrical fragments of pastes molded in a disposable plastic straw.
% % Analyses were performed at magnifications of 500×, 2000×, and 5000×, with an acceleration voltage of 15 kV.

