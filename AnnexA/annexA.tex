\label{annex:solid_precursors}

This annex summarizes the characteristics of common materials that can be used as solid precursors in the production of more sustainable binders.

\begin{landscape}
\begin{table}[p]
  \centering
  \caption{Characteristics of common or innovative residual materials that can be added to concrete to produce more sustainable binders \cite{Nodehi2021}.}
  \vspace{0.5cm}
  {\small % Reduce font size for the table
  \renewcommand{\arraystretch}{1.2} % Increase row spacing
  % \begin{tabular}{p{3.5cm} p{3cm} p{2cm} p{2cm} p{4cm} p{6cm}}
  \begin{tabular}{p{2cm} p{2.75cm} p{2cm} p{2cm} p{4cm} p{6cm}}
    \hline
    Additive name & Usual form & Average density (kg/m\textsuperscript{3}) & Average particle size (\textmu m) & Limitations & Benefits \\
    \hline
    Silica fume & Spherical & 2200 & 0.1--0.5 & Reduces workability and initial strength & Increases compactness, mechanical strength, and durability \\
    GGBFS & Angular with rough surface & 1000--1300 & 1.25--250 & Low initial strength & Increases durability, improves ITZ, and sulfate resistance \\
    Fly ash & Spherical & 540--860 & 0.5--300 & Low initial strength & Improves workability and long-term strength \\
    Metakaolin & Porous, lamellar, and angular & 890 & 1--20 & Reduces workability & Fills microstructure and improves ITZ \\
    Rice husk ash & Irregular with high porosity & 504--700 & 5--10 & Property variation and low reactivity & High silica content; improves compactness and strength \\
    Glass powder & Irregular & 2500 & 0.8--50 & High contamination & Improves durability and pozzolanic reaction \\
    Red mud & Irregular and needle-shaped & 2700--3400 & 100 to over 200 & High contamination & High alumina content, can improve hydration \\
    Ceramic waste & Angular & ~1700 & Below 100 & -- & Improves compactness and performance \\
    MSWI & Irregular & 660--1690 & -- & -- & Improves microstructure and reduces porosity \\
    Paper sludge ash & Irregular & Below 100 & -- & -- & Favourably adjusts the S/A ratio \\
    \hline
  \end{tabular}
  }
\label{tab:common_precursors}
\end{table}
\end{landscape}
