This study successfully developed a one-part alkali-activated cement system composed of low-calcium solid precursors (MK and SF) and alternative alkaline sources (K\textsubscript{2}CO\textsubscript{3} and Ca(OH)\textsubscript{2}).
The results demonstrated that the proposed formulation is technically viable, producing stable and compact matrices with adequate mechanical performance and improved safety and sustainability compared to conventional alkali-activated systems.

The variation of the Si/Al molar ratio directly influenced both the polymerization kinetics and the microstructural characteristics of the binders.
Mixtures with intermediate Si/Al ratios (particularly around 3.0) achieved the best balance between density, porosity, and mechanical strength, reflecting an optimal degree of geopolymerization and a more homogeneous gel formation.
Samples with low Si/Al ratios showed the presence of unreacted particles and larger pore sizes, whereas highly siliceous formulations tended to exhibit excessive viscosity, hindering workability and leading to heterogeneous matrices.

The microstructural analyses (XRD, FTIR and SEM/EDS) confirmed the formation of predominantly amorphous aluminosilicate gels (N-A-S-H),
where the gel characteristics and composition depended on the precursor proportion.
MIP and He pycnometry observed a pore refinement and total porosity reduction, indicating improved matrix structure.
This results reflected in the compressive strength tests, which indicated a significant increase over time due to the continuous condensation of silicate and aluminate species.

Overall, the study demonstrates that combining MK and SF with mild alkaline sources is a promising strategy for developing safer and more environmentally benign binders.
The use of K\textsubscript{2}CO\textsubscript{3} and Ca(OH)\textsubscript{2} effectively activated the aluminosilicate network, achieving mechanical and microstructural properties comparable to those of more hazardous systems based on NaOH or sodium silicate.

Future work should focus on expanding the range of solid precursors to include natural and locally available aluminosilicate materials with similar chemical composition to metakaolin and silica fume, such as calcined clays, volcanic tuffs, or agricultural ashes.
Exploring the synergy between these materials and alternative activators under different curing conditions will provide valuable insights into optimizing performance and ensuring large-scale applicability.
Additionally, long-term durability studies under variable environmental exposures are recommended to validate the practical potential of these one-part alkali-activated systems for civil construction.