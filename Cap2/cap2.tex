\section{Definition and Reaction Mechanism of Alkali-Activated Materials}

Alkali-activated materials (AAMs) are a class of inorganic materials produced by the reaction of an aluminosilicate precursor with an alkaline source under controlled conditions, forming a hardened matrix similar or superior to ordinary Portland cement (OPC).
Within this family, geopolymers represent a subclass characterized by low calcium content and the predominance of amorphous sodium or potassium aluminosilicate hydrate gels (N-A-S-H) \cite{provis2018alkali}.  
The distinction between them lies in both composition and reaction mechanisms, as depicted in Figure \ref{fig:al_ca_aam}.

\begin{figure}[htb!]
  \centering
  \includegraphics[width=0.5\textwidth]{Cap2/images/al_ca_aam.png}
  \caption{Classification of different subsets of alkali-activated materials, with comparisons to Portland cement and
calcium sulfoaluminate binder chemistry. Shading
indicates approximate alkali content; darker shading
corresponds to higher concentrations of Na and/or K \cite{rakhimova2019reaction}.}
  \label{fig:al_ca_aam}
\end{figure}

The chemical mechanism involves three main stages: (i) dissolution of the aluminosilicate framework of the precursor under highly alkaline conditions, releasing silicate and aluminate species; (ii) reorganization and condensation of these dissolved species into chains; and (iii) polycondensation forming a three-dimensional gel network that subsequently hardens into a solid binder \cite{Severo2013,duxson2006geopolymer}. Figure \ref{fig:activation} illustrates this process.

\begin{figure}[htb!]
  \centering
  \includegraphics[width=0.75\textwidth]{Cap2/images/activation.png}
  \caption{Scheme of the alkaline activation process \cite{duxson2006geopolymer}.}
  \label{fig:activation}
\end{figure}

The resulting gel composition-typically N-A-S-H for low-calcium precursors and C-A-S-H for high-calcium precursors-depends on both the chemistry of the starting materials and the activation conditions.  


\section{Historical Context}

The synthesis of materials by alkali activation began in the 1930s and 1940s, when a substitute for traditional Portland cement was developed from blast furnace slag and other aluminosilicates \cite{pachecotorgal2014handbook}.
From the 1970s onwards, interest in this area increased, when the French scientist Joseph Davidovits coined the term "geopolymer" and patented several formulations. His initial studies focused on the development of inorganic, non-flammable, and fire-resistant materials \cite{provis2009geopolymers}.

Since then, alkali-activated materials have attracted the attention of researchers and industry due to their low energy consumption and sustainable nature \cite{qin2022onepart}.
Furthermore, as studies have advanced, AAMs have gained recognition for their mechanical properties and durability, as the polymerization reactions that occur during curing provide high compressive strength and resistance to chemical attack.

\section{Raw Materials for Alkali-Activated Systems}

\subsection{Precursors}


Precursors are aluminosilicate-rich solids that react with alkaline activators to form polymeric gels. They are generally categorized into three main groups depending on their calcium oxide content and the dominant hydration products \cite{luukkonen2017review,pachecotorgal2014handbook}.


Figure \ref{fig:ternary_diagram} shows the most common precursors and their respective chemical compositions. For further details, Annex \ref{annex:solid_precursors} presents their main characteristic, limitations, and benefits.

\begin{figure}[htb!]
  \centering
  \includegraphics[width=0.625\textwidth]{Cap2/images/ternary_diagram.png}
  \caption{Ternary diagram of the most common precursors \cite{giergiczny2019fly}.}
  \label{fig:ternary_diagram}
\end{figure}

\subsubsection{Low-calcium systems}

The most common materials are industrial by-products like Class F fly ash (FA) or thermally activated natural resources such as calcined clays, like metakaolin (MK). The predominant products are N-A-S-H gels, which structure is amorphous, highly cross-linked, and chemically resembles a disordered zeolite, offering superior thermal stability, chemical resistance, and reduced shrinkage, though typically requiring longer curing time and temperature between 80-100 $\degree$C to reach the appropriate mechanical strength \cite{Provis2014_LowCa, provis2014geopolymers,ke2021one, Nodehi2021}.

\subsubsection{High-calcium systems}

High-calcium AAMs are derived from precursors rich in CaO, making them chemically closer to traditional hydraulic cements.
The main precursors are materials such as ground granulated blast furnace slag (GGBFS).
The primary reaction product is an aluminum-substituted calcium silicate hydrate (C-A-S-H) gel, which is structurally similar to the C-S-H found in Portland cement but incorporates aluminum in its structure.
The C-A-S-H gel formed in alkali-activated GGBFS typically has a lower Ca/Si ratio than hydrated Portland cement \cite{Provis2014_LowCa,PolSegura2023}.
These systems exhibit rapid strength development and good early-age performance but are more susceptible to shrinkage, cracking, and corrosion due to chloride attack.


\subsubsection{Hybrid systems}

Mixtures containing both calcium and aluminosilicate-rich materials, such as blast furnace slag (BFS) combined with FA or MK, or systems containing low levels of Portland cement, where both C-A-S-H and N-A-S-H phases coexist, potentially optimizing mechanical strength and durability \cite{pachecotorgal2014handbook, luukkonen2017review,Provis2014_LowCa}.


% \subsubsection{Metakaolin}

% \subsubsection{Silica Fume}

\subsection{Activators}

The alkaline attack on the microstructure of the precursors results in the release of silicates and aluminates into the solution.
Hence, the solubility of silica and alumina is important for studies in this field, which can be expressed as a function of pH (Figure \ref{fig:solubility}).

\begin{figure}[htb!]
  \centering
  \includegraphics[width=0.625\textwidth]{Cap2/images/solubility.png}
  \caption{Solubility of silica and alumina as a function of pH \cite{mason1952principles}.}
  \label{fig:solubility}
\end{figure}

It is observed that the solubility of silica is low in acidic environments and high in basic media, while alumina is soluble at both extremes of pH.
Therefore, for the activation reactions to occur, it is necessary that the pH of the solution is high.
The main alkaline activators are: sodium hydroxide (NaOH), sodium silicate (Na\textsubscript{2}SiO\textsubscript{3}), potassium hydroxide (KOH), sodium carbonate (Na\textsubscript{2}CO\textsubscript{3}) and potassium carbonate (K\textsubscript{2}CO\textsubscript{3}).

Alkaline activators can form two different systems: (i) one-part systems are designed as dry powder mixtures incorporating the precursor and a solid alkaline activator, requiring only the addition of water ("just-add-water"), this approach enables factory production, bagging, and distribution analogous to conventional cement \cite{provis2018alkali}; and (ii) two-part systems, where liquid alkaline solutions are prepared separately and then mixed with the precursor

\section{Current Limitations and Research Gaps}
Despite the numerous advantages of alkali-activated materials, several challenges hinder their widespread adoption in the construction industry.

\subsection{Safety Concerns}

The first studies on AAMs focused on two-part systems, since the final product exhibits high compressive strength, adhesion, and the ability to withstand fatigue loads.
In addition, they also demonstrate high resistance to freeze-thaw cycles and high temperatures \cite{heath2014gwp}.

However, two-part systems require the mixing of a solid precursor with a concentrated aqueous alkaline solution at the point of use. These solutions are corrosive and irritate human skin, making their transport and handling hazardous for workers \cite{awoyera2019critical}.

The need to handle highly corrosive, viscous, and often hazardous activator solutions in the field is a major impediment to their practicality, mixing, and large-scale market adoption \cite{Shah2020}.

% Consequently, to enable the broader and safer application of one-part alkali-activated binders, it is essential to seek alternative alkaline sources that exhibit lower corrosivity while maintaining sufficient reactivity for effective activation.

\subsection{Alternative sources}

The availability and long-term supply of conventional industrial by-products remain a challenge.
GGBFS and FA are the most popular precursors, with GGBFS often yielding the highest mechanical strength \cite{qin2022onepart}.
However, the supply of FA is projected to decrease globally due to the shift away from coal energy, and BFS availability depends on local availability from the iron-making industry \cite{pachecotorgal2014handbook}.
This necessitates extensive research focused on optimizing mixes using precursors other than fly ash and blast furnace slag.
For systems targeting low-calcium AAMs, this means exploring less conventional, locally available low-Ca aluminosilicate solid waste resources \cite{rakhimova2022blended}.

\subsection{Environmental impacts}

The main environmental advantage attributed to AAMs lies in the considerable reduction of CO\textsubscript{2} emissions compared to traditional Portland cement.
It is estimated that the environmental impacts of solid and liquid activators are 24\% and 60\% of the impact caused by OPC, respectively \cite{luukkonen2017review}.

% Furthermore, the production of AAMs often utilizes industrial residues as raw materials, such as sewage sludge ash, sugarcane straw ash, among others \cite{moraes2024scsa}.
% Therefore, in addition to the valorization of industrial waste, the production of AAMs reduces the demand for natural resources from mineral deposits and provides an environmentally appropriate destination, following the guidelines of the National Solid Waste Policy \cite{PNRS2016}.

However, the production of sodium silicate - which is the main alkaline activator used in one-part AAM with low-calcium precursors \cite{luukkonen2017review} - occurs between 1200-1400$\degree$C and emits approximately 1.514 kg of CO\textsubscript{2} per kg of silicate produced, in addition to significantly contributing to air pollution through dust and nitrogen and sulfur oxides \cite{rajan2020sustainable}.
Indeed, some studies show that the emissions profile of certain alkali-activated concrete/binders (AACB) mixtures can be worse than OPC if synthetic activators dominate the material composition \cite{pachecotorgal2014handbook}.

Therefore, a crucial research gap lies in the pursuit of alternative and environmentally friendly solid alkaline sources that are sodium silicate free or require minimal synthetic input \cite{luukkonen2017review}.

\section{Justification of the present research}

This research directly addresses the limitations in three ways: (i) practicality; (ii) precursor diversity; and (iii) environmental impact.

By developing a one-part mix system, this study eliminates the need for handling corrosive liquid alkaline activators, promoting user-friendly product commercialization and simplifying logistics for construction applications.

By utilizing low-calcium solid precursors (instead of high-calcium), the research contributes to diversifying the raw materials base and reducing reliance on traditional industrial by-products whose availability is increasingly strained or declining.

% The core objective is to investigate and integrate alternative alkaline sources to replace or significantly minimize the use of high-CO\textsubscript{2} footprint synthetic sodium silicate.
By concentrating on non-hazardous and less polluting solid alkaline sources, this work aims to develop an environmentally and economically competitive binder, contributing significantly to the reduction of the construction industry's carbon and ecological footprint.

In essence, this work provides necessary scientific investigation into optimizing the mix design for the next generation of alkali-activated binders: cost-effective, easily deployable (one-part), and truly sustainable systems based on low-calcium precursors and eco-friendly activation agents.
