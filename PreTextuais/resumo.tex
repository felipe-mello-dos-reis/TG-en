Na busca por alternativas mais sustentáveis ao cimento Portland, os cimentos ativados
alcalinamente têm sido amplamente estudados. No entanto, a maioria dos processos
de mistura ocorre em duas etapas, o que impacta a viabilidade e a eficiência
construtiva. Um avanço importante nessa área é o desenvolvimento de sistemas
monocomponentes, que simplificam a produção e tornam a tecnologia mais acessível
e prática. Ainda assim, os estudos atuais se concentram em precursores ricos em
cálcio, enquanto o uso de fontes alcalinas tradicionais apresenta desafios relacionados
à segurança e ao custo. Este trabalho propõe o desenvolvimento de um cimento
ativado alcalinamente monocomponente utilizando precursores sólidos de baixo teor
de cálcio, como sílica ativa e metacaulim, e fontes alcalinas mais seguras e acessíveis,
como carbonato de potássio e hidróxido de cálcio, garantindo resistência mecânica
adequada e maior viabilidade para aplicação na construção civil.