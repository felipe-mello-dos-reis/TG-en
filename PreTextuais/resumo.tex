Na busca por alternativas mais sustentáveis ao cimento Portland, os cimentos ativados alcalinamente têm sido amplamente estudados.
Inicialmente, a maioria dos processos de mistura ocorre em duas etapas, que sacrificam a eficiência produtiva em função das melhores proriedades mecânicas.
Com o objetivo  aumentar a escalabilidade do processo, o desenvolvimento de sistemas monocomponentes trouxe uma tecnologia mais acessível e prática para a indústria.
Ainda assim, os estudos atuais se concentram em precursores ricos em
cálcio, enquanto o uso de fontes alcalinas tradicionais apresenta desafios relacionados à segurança e ao custo.
Este trabalho propõe o desenvolvimento de um cimento ativado alcalinamente monocomponente utilizando precursores sólidos de baixo teor de cálcio, como sílica ativa e metacaulim, e fontes alcalinas mais seguras e acessíveis, como carbonato de potássio e hidróxido de cálcio, garantindo resistência mecânica adequada e maior viabilidade para aplicação na construção civil.