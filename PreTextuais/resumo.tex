Na busca por alternativas mais sustentáveis ao cimento Portland, os cimentos ativados alcalinamente têm sido amplamente estudados.
Inicialmente, a maioria dos processos de mistura ocorre em duas etapas, que sacrificam a eficiência produtiva em favor de melhores propriedades mecânicas.
Visando aumentar a escalabilidade do processo, o desenvolvimento de sistemas monocomponentes (``just-add-water") trouxe uma tecnologia mais acessível e prática para a indústria.
No entanto, os estudos atuais se concentram principalmente em precursores ricos em cálcio, enquanto o uso de ativadores alcalinos convencionais levanta preocupações relacionadas à segurança e ao impacto ambiental.
Este trabalho propõe o desenvolvimento de um cimento ativado alcalinamente monocomponente usando precursores sólidos de baixo teor de cálcio, como sílica ativa e metacaulim, juntamente com fontes alcalinas mais seguras e sustentáveis, como carbonato de potássio e hidróxido de cálcio.
A variação da razão molar Si/Al influenciou diretamente tanto a cinética de polimerização quanto as características microestruturais dos aglomerantes.
Misturas com Si/Al = 3,0 alcançaram o melhor equilíbrio entre densidade, porosidade e resistência mecânica, refletindo um grau ótimo de geopolimerização e formação homogênea de gel.
As análises microestruturais confirmaram a formação de géis aluminossilicatos predominantemente amorfos (K-A-S-H), com características do gel dependentes das proporções dos precursores.
Os resultados demonstraram que a formulação proposta é tecnicamente viável, produzindo matrizes estáveis e compactas com desempenho mecânico adequado e maior segurança e sustentabilidade em comparação com sistemas ativados alcalinamente convencionais.
O estudo demonstra que a combinação de metacaulim e sílica ativa com fontes alcalinas suaves é uma estratégia promissora para desenvolver um aglomerante com desempenho mecânico adequado e viabilidade aprimorada para aplicação na indústria da construção.
