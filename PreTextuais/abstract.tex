In the search for more sustainable alternatives to Portland cement, alkali-activated cements have been extensively studied.
Initially, most mixing processes occur in two steps, which sacrifice productive efficiency in favor of improved mechanical properties.
Aiming to increase process scalability, the development of one-part (just-add-water) systems has brought a more accessible and practical technology to the industry.
Nonetheless, current studies mainly focus on calcium-rich precursors, while the use of conventional alkaline activators raises concerns related to safety and environmental impact.
This work proposes the development of a one-part alkali-activated cement using low-calcium solid precursors, such as silica fume and metakaolin, along with safer and sustainable alkaline sources, such as potassium carbonate and calcium hydroxide.
The variation of the Si/Al molar ratio directly influenced both the polymerization kinetics and microstructural characteristics of the binders.
Mixtures with Si/Al = 3.0 achieved the best balance between density, porosity, and mechanical strength, reflecting an optimal degree of geopolymerization and homogeneous gel formation.
Microstructural analyses confirmed the formation of predominantly amorphous aluminosilicate gels (K-A-S-H), with gel characteristics dependent on precursor proportions.
The results demonstrated that the proposed formulation is technically viable, producing stable and compact matrices with adequate mechanical performance and improved safety and sustainability compared to conventional alkali-activated systems.
The study demonstrates that combining metakaolin and silica fume with mild alkaline sources is a promising strategy for developing a binder with adequate mechanical performance and enhanced viability for application in the construction industry.