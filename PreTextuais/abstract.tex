In the search for more sustainable alternatives to Portland cement, alkali-activated cements have been extensively studied.
Initially, most mixing processes occur in two steps, which sacrifice productive efficiency in favor of improved mechanical properties.
Aiming to increase process scalability, the development of one-part (just-add-water) systems has brought a more accessible and practical technology to the industry.
Nonetheless, current studies mainly focus on calcium-rich precursors, while the use of conventional alkaline activators raises concerns related to safety and cost.
This work proposes the development of a one-part alkali-activated cement using low-calcium solid precursors, such as silica fume and metakaolin, along with safer and more affordable alkaline sources, such as potassium carbonate and calcium hydroxide.
The goal is to ensure adequate mechanical performance while enhancing the viability of these materials for application in the construction industry.