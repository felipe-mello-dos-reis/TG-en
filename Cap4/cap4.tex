\section{Chemical and Physical Characterization of Precursors}
\label{sec:chemical_and_physical_characterization_of_precursors}

\subsection{Chemical Composition of Precursors}
\label{sec:chemical_composition_of_precursors}

The chemical composition of the aluminosilicate precursors plays a fundamental role in determining their reactivity and suitability for activation in a low-calcium system. 
In this work, the metakaolin (MK) and active silica (SA) precursors were analysed for their oxide content (normalized with respect to the powder fraction) and loss on ignition (LOI), which was found to be 0.69\% and 2.27\%, respectively.
% The results are presented in Table \ref{tab:mk_sa_composition}.

\begin{table}[H]
    \centering
    \caption{Chemical composition (wt \%) of the precursors: metakaolin (MK) and active silica (SA).}
    \label{tab:mk_sa_composition}
    \begin{tabular}{l c c c c}
        \hline
        \multicolumn{1}{c}{Oxide} & \multicolumn{2}{c}{Metakaolin} & \multicolumn{2}{c}{Active Silica} \\
        \cline{2-5}
        & wt (\%) & wt with LOI (\%) & wt (\%) & wt with LOI (\%) \\
        \hline
        K\textsubscript{2}O & 0.21 & 0.20 & 0.74 & 0.73 \\
        CaO & 0.19  & 0.19 & 0.13 & 0.13 \\
        MgO & 2.60 & 2.59 & 0.00 & 0.00 \\
        P\textsubscript{2}O\textsubscript{5} & 0.00 & 0.00 & 0.00 & 0.00 \\
        Cl & 0.00 & 0.00 & 0.12 & 0.12 \\
        SO\textsubscript{3} & 0.00 & 0.00 & 0.15 & 0.15 \\
        SiO\textsubscript{2} & 49.37 & 49.03 & 96.90 & 94.70 \\
        Fe\textsubscript{2}O\textsubscript{3} & 0.77 & 0.76 & 1.78 & 1.74 \\
        Al\textsubscript{2}O\textsubscript{3} & 46.38 & 46.06 & 0.00 & 0.00 \\
        Na\textsubscript{2}O & 0.00 & 0.00 & 0.17 & 0.17 \\
        TiO\textsubscript{2} & 0.49 & 0.48 & 0.00 & 0.00 \\
        \hline
    \end{tabular}
\end{table}

From Table 4.1.1 it is clear that the MK precursor is rich in alumina and contains a significant silica fraction , while the SA is extremely high in silica and essentially alumina-free.
Critically, both materials present very low CaO contents.
The minimal presence of calcium oxide is an important indicator of the low-calcium nature of the precursors, which is a pre-requisite for favoring the formation of potassium aluminosilicate hydrate type gels, rather than calcium-rich gels, as often observed when using ground granulated blast furnace slag \cite{ali2023geopolymer}.


In addition, the low LOI values suggest a limited amount of residual organics or volatile components, which could otherwise interfere with the dissolution kinetics of the aluminosilicates.
Furthermore, the Si/Al molar ratio of MK precursos is approximately 0.9, which will be used with the essentially pure SA to tailor the mix designs for optimal geopolymerization reactions, as presented in Appendix \ref{appendix:mix_designs}.

% Create a slide with this phrase:
In summary, the chemical data confirm that both MK and SA meet the key requirements of (i) high silica and/or alumina content, (ii) low calcium content, and (iii) minimal impurities, thereby validating their use as raw materials for a low calcium alkali-activated binder system.

\subsection{X-Ray Diffraction of Precursors and Alkaline Sources}
\label{sec:x-ray_diffraction_of_precursors_and_alkaline_sources}

XRD analysis was conducted on the precursors and alkaline sources to evaluate phase composition, crystallinity and the presence of amorphous phases. The key observations were as follows:

\begin{table}[H]
    \centering
    \caption{Summary of XRD phase observations for precursors and alkaline sources.}
    \label{tab:precursor_xrd_summary}
    \begin{tabular}{l l}
        \hline
        Material & Key XRD observations \\
        \hline
        MK & Broad amorphous hump near $2\theta \approx 21\degree$, residual quartz peaks (COD 900-9667) \\
        SA & Broad amorphous hump near $2\theta \approx 22.5\degree$, no distinct crystalline phases \\
        Ca(OH)\textsubscript{2} & Crystalline portlandite phase (COD 900-0114) \\
        K\textsubscript{2}CO\textsubscript{3} & Crystalline calcium carbonate phase (COD 900-9644) \\
        \hline
    \end{tabular}
\end{table}

SA and MK both exhibited broad amorphous humps in their XRD patterns, indicative of their largely non-crystalline nature, which is favorable for dissolution during alkali activation.
The MK showed minor crystalline quartz peaks, indicating the presence of impurities, which remains inert during calcination \cite{provis2014geopolymers} and acts as filler without participating in geopolymerization reaction \cite{rakhimova2019metakaolin}.
The alkaline sources displayed sharp diffraction peaks corresponding to their known crystalline phases, confirming their purity.

% \section{}

%% AGLOMERANTE ÁLCALI-ATIVADO DE PARTE ÚNICA: OBTENÇÃO, COMPOSIÇÃO, PROPRIEDADES E DURABILIDADE
% Os resultados mostram que as pastas apresentaram fases semelhantes. As fases caulinita (Al2O3·2SiO2·2H2O, COD ID: 9009234), muscovita (KAl3Si3O10(OH)2, COD ID: 9005014) e quartzo (SiO2, ICSD PDF: 01-085-0457), presentes nas argilas precursoras, permaneceram presentes nas pastas, tal qual observado por Zhang et al. (2019) em difratogramas de AAA à base metacaulim. Isso acontece porque essas fases cristalinas são pouco afetadas pelo ataque alcalino (RUIZ-SANTAQUITERIA et al., 2012).
% Estudo de Melo et al. (2017) indicou que as fases cristalinas de quartzo e muscovita presentes no metacaulim não participam das reações químicas formadoras do aglomerante. Entretanto, os resultados indicam que houve diminuição na intensidade de alguns picos em comparação com o padrão do metacaulim precursor, o que pode sugerir certa participação das fases cristalinas nas reações dos produtos originados (ZHANG et al., 2012). Os halos observados nas pastas entre 2θ = 18° a 35° também apareceram nos difratogramas dos metacaulins precursores, sendo que esses halos indicam a presença de fases amorfas nas misturas (RUIZ-SANTAQUITERIA et al., 2012; MEJÍA et al., 2016). Comumente, o gel aluminossilicato N-A-S-H apresenta um halo amorfo entre 2θ = 20° a 35° (GOMEZ-ZAMORANO et al., 2017).